% https://trac.ffmpeg.org/wiki

\title{\href{https://www.ffmpeg.org}{FFMPEG}/\href{https://libav.org}{Libav}}

\maketitle
\tableofcontents

\chapter{Introduction}
%{{{

\section{FFMPEG vs Libav}
%{{{

\begin{itemize}
\item Libav is a fork of FFMPEG which was originally started by
  Fabrice Bellard. The fork was basically
  a \href{http://blog.pkh.me/p/13-the-ffmpeg-libav-situation.html}{non-amicable
  result of conflicting personalities and development styles within
  the FFmpeg community}.
\end{itemize}

%}}}

\section{Free sofware and patents}
%{{{

\begin{itemize}
\item Both, FFMPEG and Libav contains more than 100 codecs. Many
  codecs that compress information have been claimed by patent
  holders. Such claims may be enforceable in countries like the United
  States which have implemented software patents, but are considered
  unenforceable or void in countries that have not implemented
  patents.
\end{itemize}

%}}}

%}}}

\chapter{\href{https://trac.ffmpeg.org/wiki/CompilationGuide}{Installation}}
%{{{

\section{FFMPEG}
%{{{

\subsection{\href{https://trac.ffmpeg.org/wiki/CompilationGuide/Ubuntu}{Ubuntu}}
%{{{

\begin{itemize}
\item In Debian/Ubuntu this is the preferable way because: (1) is the most general --
  it could be used in any distribution without significative
  modifications --, (2) uses the lastest software and (3) produces the
  most efficient machine code.
\begin{verbatim}
mkdir ~/src
cd ~/src

# libx264
wget http://download.videolan.org/pub/x264/snapshots/last_x264.tar.bz2
tar xjvf last_x264.tar.bz2
cd x264-snapshot*
PATH="$PATH:$HOME/bin" ./configure --prefix="$HOME/ffmpeg_build" --bindir="$HOME/bin" --enable-static --disable-opencl
PATH="$PATH:$HOME/bin" make
make install
make distclean
cd ..

# libfdk-aac (make sure "libtool" is installed)
wget -O fdk-aac.zip https://github.com/mstorsjo/fdk-aac/zipball/master
unzip fdk-aac.zip
cd mstorsjo-fdk-aac*
autoreconf -fiv
./configure --prefix="$HOME/ffmpeg_build" --disable-shared
make
make install
make distclean
cd ..

# libmp3lame (make sure "nasm" is installed)
wget http://downloads.sourceforge.net/project/lame/lame/3.99/lame-3.99.5.tar.gz
tar xzvf lame-3.99.5.tar.gz
cd lame-3.99.5
./configure --prefix="$HOME/ffmpeg_build" --enable-nasm --disable-shared
make
make install
make distclean
cd ..

# libopus
wget http://downloads.xiph.org/releases/opus/opus-1.1.tar.gz
tar xzvf opus-1.1.tar.gz
cd opus-1.1
./configure --prefix="$HOME/ffmpeg_build" --disable-shared
make
make install
make distclean
cd ..

# libvpx (VP8/VP9)
wget http://webm.googlecode.com/files/libvpx-v1.3.0.tar.bz2
tar xjvf libvpx-v1.3.0.tar.bz2
cd libvpx-v1.3.0
PATH="$PATH:$HOME/bin" ./configure --prefix="$HOME/ffmpeg_build" --disable-examples
PATH="$PATH:$HOME/bin" make
make install
make clean
cd ..

# libogg
wget http://downloads.xiph.org/releases/ogg/libogg-1.3.1.tar.gz
tar xvf libogg-1.3.1.tar.gz
cd libogg-1.3.1
./configure --prefix="$HOME/ffmpeg_build" --disable-shared
make
make install
make distclean
cd ..

# libvorbis
wget http://downloads.xiph.org/releases/vorbis/libvorbis-1.3.3.tar.gz
tar xvf libvorbis-1.3.3.tar.gz
cd libvorbis-1.3.3
./configure --prefix="$HOME/ffmpeg_build" --disable-shared
make
make install
make distclean
cd ..

# libtheora (make sure "texlive-fonts-recommended" and "doxygen" are installed)
wget http://downloads.xiph.org/releases/theora/libtheora-1.1.1.tar.bz2
tar xjvf libtheora-1.1.1.tar.bz2
cd libtheora-1.1.1
./configure --prefix="$HOME/ffmpeg_build" --disable-shared --with-ogg="$HOME/ffmpeg_build"
make
make install
make distclean
cd ..

# ffmpeg (make sure "libass-dev", "libasound2-dev", "libsdl1.2-dev" 
# and "libx11-dev" are installed)
git clone git://source.ffmpeg.org/ffmpeg.git ffmpeg
cd ffmpeg
PATH="$PATH:$HOME/bin" PKG_CONFIG_PATH="$HOME/ffmpeg_build/lib/pkgconfig" ./configure \
  --prefix="$HOME/ffmpeg_build" \
  --extra-cflags="-I$HOME/ffmpeg_build/include" \
  --extra-ldflags="-L$HOME/ffmpeg_build/lib" \
  --bindir="$HOME/bin" \
  --enable-gpl \
  --enable-libass \
  --enable-libfdk-aac \
  --enable-libfreetype \
  --enable-libmp3lame \
  --enable-libopus \
  --enable-libtheora \
  --enable-libvorbis \
  --enable-libvpx \
  --enable-libx264 \
  --enable-nonfree \
  --extra-libs=-lasound \
  --enable-x11grab

# Note: at the moment of writting this document, there is a bug in the
# "x11grab" code (libxcb). If it persits, remove this option.

make
make install
make distclean
hash -r
\end{verbatim}
\end{itemize}

%}}}

\subsection{\href{https://wiki.archlinux.org/index.php/FFmpeg}{Arch Linux}}
%{{{

\begin{itemize}
\item Standard package:
%{{{

\begin{verbatim}
sudo pacman -S ffmpeg
\end{verbatim}

%}}}

\item Full package:
% Instalar en el arch linux del mac
\begin{verbatim}
wget https://aur.archlinux.org/packages/ff/ffmpeg-full/ffmpeg-full.tar.gz
tar xzvf ffmpeg-full.tar.gz
cd ffmpeg-full
makepkg # We see that there are several dependiencies
cd ..

# From the official repositories:
sudo pacman -S yasm ladspa hardening-wrapper

# Installing shine from AUR:
wget https://aur.archlinux.org/packages/sh/shine/shine.tar.gz
tar xzvf shine.tar.gz
cd shine
makepkg
sudo pacman -U shine-3.1.0-1-i686.pkg.tar.xz
cd ..

# Installing libbs2b from AUR:
wget https://aur.archlinux.org/packages/li/libbs2b/libbs2b.tar.gz
tar xzvf libbs2b.tar.gz
cd libbs2b
makepkg
sudo pacman -U libbs2b-3.1.0-2-i686.pkg.tar.xz
cd ..

# Installing opencl-headers12 from AUR:
wget https://aur.archlinux.org/packages/op/opencl-headers12/opencl-headers12.tar.gz
tar xzvf opencl-headers12.tar.gz
cd opencl-headers12
makepkg
sudo pacman -U opencl-headers12-1\:1.2.r29463-1-any.pkg.tar.xz
cd ..

# Installing utvideo-git from AUR:
wget https://aur.archlinux.org/packages/ut/utvideo-git/utvideo-git.tar.gz
tar xzvf utvideo-git.tar.gz 
cd utvideo-git/
sudo pacman -S nasm
makepkg
sudo pacman -U utvideo-git-15.0.0_87_6135aba-1-i686.pkg.tar.xz
cd ..

# Installing vid.stab from the official repositories:
sudo pacman -S vid.stab

# Installing frei0r-plugins from the official repositories:
sudo pacman -S frei0r-plugins

# Installing zeromq from the official repositories:
sudo pacman -S zeromq

# Installing wavpack from the official repositories:
sudo pacman -S wavpack

# Installing vo-amrwbenc from AUR:
wget https://aur.archlinux.org/packages/vo/vo-amrwbenc/vo-amrwbenc.tar.gz
tar xzvf vo-amrwbenc.tar.gz
cd vo-amrwbenc
makepkg
sudo pacman -U vo-amrwbenc-0.1.3-1-i686.pkg.tar.xz
cd ..

# Installing vo-aacenc from AUR:
wget https://aur.archlinux.org/packages/vo/vo-aacenc/vo-aacenc.tar.gz
tar xzvf vo-aacenc.tar.gz
cd vo-aacenc
makepkg
sudo pacman -U vo-aacenc-0.1.3-1-i686.pkg.tar.xz
cd ..

# Installing twolame from the official repositories:
sudo pacman -S twolame

# Installing openal from the official repositories:
sudo pacman -S openal

# Installing libwebp from the official repositories:
sudo pacman -S libwebp

# Installing libssh from the official repositories:
sudo pacman -S libssh

# Installing libiec61883 from the official repositories:
sudo pacman -S libiec61883

# Installing libgme from the official repositories:
sudo pacman -S libgme

# Installing libfdk-aac from the official repositories:
sudo pacman -S libfdk-aac

# Installing  from the official repositories:
sudo pacman -S libavc1394

# Installing libaacplus from AUR:
wget https://aur.archlinux.org/packages/li/libaacplus/libaacplus.tar.gz
tar xzvf libaacplus.tar.gz
cd libaacplus
sudo pacman -S unzip
makepkg
sudo pacman -U libaacplus-2.0.2-8-i686.pkg.tar.xz
cd ..

# Installing decklink-sdk from AUR:
wget https://aur.archlinux.org/packages/de/decklink-sdk/decklink-sdk.tar.gz
tar xzvf decklink-sdk.tar.gz
cd decklink-sdk/
makepkg
sudo pacman -U decklink-sdk-1\:10.1.4-1-i686.pkg.tar.xz
cd ..

# Installing celt from the official repositories:
sudo pacman -S celt

# Installing libsoxr from AUR:
wget https://aur.archlinux.org/packages/li/libsoxr/libsoxr.tar.gz
tar xzvf libsoxr.tar.gz
cd libsoxr/
sudo pacman -S cmake
makepkg
sudo pacman -U libsoxr-0.1.1-2-i686.pkg.tar.xz
cd ..

Edit your ~/.gnupg/gpg.conf and uncomment the following line:
keyserver-options auto-key-retrieve

cd ffmpeg-full
makepkg
sudo pacman -U ffmpeg-full-1\:2.5.3-1-i686.pkg.tar.xz
\end{verbatim}

\end{itemize}

%}}}

%}}}

\section{Libav}
%{{{

\subsection{Ubuntu}
%{{{

\begin{verbatim}
sudo apt-get install x264           # H.264
sudo apt-get install libav-tools    # avconv
\end{verbatim}

%}}}

%}}}

%}}}

\chapter{Tools and help}
%{{{

\section{FFMPEG/Libav command line tools}
%{{{

\begin{enumerate}
\item \href{https://www.ffmpeg.org/ffmpeg.html}{\texttt{ffmpeg}}/\texttt{avconv}: encodes and transcodes.
\item \href{https://www.ffmpeg.org/ffplay.html}{\texttt{ffplay}}/\texttt{avplay}: playbacks a video.
\item \href{https://www.ffmpeg.org/ffprobe.html}{\texttt{ffprobe}}/\texttt{avprobe}: gathers information from multimedia streams.
\item \href{https://www.ffmpeg.org/ffserver.html}{\texttt{ffserver}}/\texttt{avserver}: a streaming server for both audio and video
\end{enumerate}

%}}}

\section{\texttt{man}ual}
%{{{

\begin{itemize}

\item For \texttt{ffmpeg} installed from sources:
%{{{

\begin{verbatim}
$ cd doc

# "man" manuals:
ls *.1
man ./ffmpeg.1      # FFMPEG's basic manual

# "HTML" manuals:
ls *.html
firefox ./ffmpeg.1
\end{verbatim}

%}}}

\item For  \texttt{ffmpeg/avconv} installed from repository:
%{{{

\begin{verbatim}
man ffmpeg
man avconv
\end{verbatim}

%}}}

\end{itemize}

%}}}

\section{Built-in help}
%{{{

\begin{verbatim}
ffmpeg -help
avconv -help
\end{verbatim}

%}}}

\section{Synopsis}
%{{{

\begin{verbatim}
ffmpeg/aconv [global_options] {[input_file_options] -i input_file} ... {[output_file_options] output_file} ... 
\end{verbatim}

%}}}

%}}}

\chapter{\href{http://www.ffmpeg.org/ffprobe.html}{File inspection}}
%{{{

\begin{verbatim}
wget http://www.hpca.ual.es/~vruiz/videos/coastguard_352x288x30x420x300.avi

ffmpeg -i /tmp/coastguard_352x288x30x420x300.avi
# or ...
ffprobe /tmp/coastguard_352x288x30x420x300.avi
\end{verbatim}

%}}}

\chapter{Capturing}
%{{{

\section{Recognized devices}
%{{{

% Revisar avconv
\begin{verbatim}
ffmpeg -devices
avconv -formats
\end{verbatim}

%}}}

\section{\href{https://trac.ffmpeg.org/wiki/Capture/ALSA}{Sound rcording (sound) from ALSA}}
%{{{

\begin{verbatim}
# Record 30 seconds, stereo, CD quality, from the ALSA mic
ffmpeg -f alsa -ac 2 -ar 44100 -i default -t 30 out.wav
ffmpeg -f alsa -ac 2 -ar 44100 -i hw:0 -t 30 out.wav
avconv -f alsa -ac 2 -ar 44100 -i hw:0 -t 30 out.wav
\end{verbatim}

%}}}

\section{\href{http://ffmpeg.org/ffmpeg-devices.html\#video4linux2_002c-v4l2}{Video recording from a V4L device}}
%{{{

\begin{itemize}

\item Video4Linux (V4L) is a part of the Linux Kernel which provides
  an unified API for video capture (such as a WebCam). At this moment,
  it is as a subsystem of the \href{http://linuxtv.org}{LinuxTV
  Project} (V4L2, Analog TV, DVB and Remote Controllers).
\end{itemize}

\begin{verbatim}
ffmpeg -f video4linux2 -s 640x480 -r 30 -i /dev/video0 /tmp/1.avi # (/dev/video0 = WebCam)
avconv -f video4linux2 -s 640x480 -r 30 -i /dev/video0 /tmp/1.avi
\end{verbatim}

%}}}

\section{\href{https://trac.ffmpeg.org/wiki/Capture/Desktop}{Video recording from the X11}}
%{{{

%\section{What is x11grab?} % Explicar y punto

\begin{verbatim}
# Grab CIF region from origin:
ffmpeg -f x11grab -framerate 25 -video_size cif -i :0.0 -f mpegts - | ffplay -
#                                                   ^ ^
#                                                   | |
#                                                   | +-----+
#                                                   |       |
#                                                  :server.display

# Grab from coordinates (10, 20):
ffmpeg -f x11grab -show_region 1 -framerate 25 -video_size cif -i :0.0+10,20 -f mpegts - | ffplay -
avconv -f x11grab -show_region 1 -framerate 25 -video_size cif -i :0.0+10,20 -f mpegts - | avplay -

# Grab following the mouse (centered in the captured window):
ffmpeg -f x11grab -show_region 1 -follow_mouse centered -framerate 25 -video_size cif -i :0.0 -f mpegts - | ffplay -
avconv -f x11grab -show_region 1 -follow_mouse centered -framerate 25 -video_size cif -i :0.0 -f mpegts - | avplay -

# Capture the full screen, with audio, and produce a file:
ffmpeg -framerate 25 -f x11grab -i :0.0 -f alsa -ac 2 -i default output.flv
\end{verbatim}

%# Follow only when the mouse pointer reaches within 100 pixels to edge:
%ffmpeg -f x11grab -show_region 1 -follow_mouse 100 -framerate 25 -video_size cif -i :0.0 -f mpegts - | ffplay -
%avconv -f x11grab -show_region 1 -follow_mouse 100 -framerate 25 -video_size cif -i :0.0 -f mpegts - | avplay -

%}}}

%}}}

\chapter{Converting}
%{{{

\section{\href{https://www.ffmpeg.org/general.html\#File-Formats}{Supported formats} and \href{https://www.ffmpeg.org/ffmpeg-codecs.html}{codecs}}
%{{{

\begin{verbatim}
ffmpeg -formats  # List formats (containers)
ffmpeg -codecs   # List codecs  (algorithms)
ffmpeg -pix_fmts # List pixel formats
\end{verbatim}

%}}}

\section{YUV}

\begin{verbatim}
# From YUV420 to Y
ffmpeg -f rawvideo -vcodec rawvideo -s 4096x4096 -r 1 -pix_fmt yuv420p -i sun_4096x4096x0.027x420x129.yuv -codec:v rawvideo -pix_fmt gray sun_4096x4096x0.027x420x129.YUV # This YUV extension is needed to put all Y images in one single file

# To show the output file:
mplayer sun_4096x4096x0.027x420x129.Y -demuxer rawvideo -rawvideo w=4096:h=4096:format=y8
mplayer sun_4096x4096x0.027x420x129.Y -demuxer rawvideo -rawvideo w=4096:h=4096:fps=1:format=y8
\end{verbatim}

%\section{MPEG-1/2}

%\section{MPEG-4}

\section{\href{http://www.videolan.org/developers/x264.html}{H.264 (x264)}}
%{{{

\begin{itemize}
\item FFMPEG/Libav does not provide a native encoder for H.264 but wraps x264.
\end{itemize}

\subsection{Presets and Tunes}
%{{{

\begin{itemize}
\item As it can be seen with:

\begin{verbatim}
x264 --help
\end{verbatim}

  presets are \texttt{ultrafast}, \texttt{uperfast},
  \texttt{veryfast}, \texttt{faster}, \texttt{fast}, \texttt{medium},
  \texttt{slow}, \texttt{slower}, \texttt{veryslow} and
  \texttt{placebo}. From left to right, the compression ratio grows
  although you more computational resources are needed (roughtly
  speaking, twice as slow as the previous). Tunings are \texttt{film},
  \texttt{animation}, \texttt{grain} (used for very grainy or old
  content), \texttt{stillimage}, \texttt{psnr} (Peak Signal-to-Noise
  Ratio), \texttt{ssim} (Structural SIMilarity), \texttt{fastdecode}
  and \texttt{zerolatency}.

\item Tunings and presets can be mixed and matched,
  e.g. -preset fast -tune film.

\end{itemize}

\subsection{Example}

\begin{verbatim}
ffmpeg -f video4linux2 -s 640x480 -r 30 -i /dev/video0 -c:v libx264 -tune zerolatency -preset medium -b:v 800k -f mpegts - | ffplay -

avconv -f video4linux2 -s 640x480 -r 30 -i /dev/video0 -c:v libx264 -tune zerolatency -preset medium -b:v 800k -f mpegts - | avplay -
\end{verbatim}

%}}}

\subsection{\href{https://trac.ffmpeg.org/wiki/Encode/H.264}{Profiles}}
%{{{

\begin{itemize}

\item Profiles enforce a standard, ensuring compatibility with
  specific targets by restricting the encoding features used. Nowadays
  most decoders support the high profile, so this setting could be
  ignored.

\end{itemize}

\begin{verbatim}
# Highest compatibility (-profile:v baseline -level 3.0)
ffmpeg -f video4linux2 -s 640x480 -r 30 -i /dev/video0 -pix_fmt yuv420p -c:v libx264 -profile:v high -level:v 4.0 -b:v 800k -f mpegts - | ffplay -

# Full compatibility (-profile:v baseline -level 30
ffmpeg -f video4linux2 -s 640x480 -r 30 -i /dev/video0 -pix_fmt yuv420p -c:v libx264 -profile:v baseline -level:v 30 -b:v 800k -f mpegts - | ffplay -
ffmpeg -f video4linux2 -s 640x480 -r 30 -i /dev/video0 -pix_fmt yuv420p -c:v libx264 -profile:v baseline -level:v 3.0 -b:v 800k -f mpegts - | ffplay -
avconv -f video4linux2 -s 640x480 -r 30 -i /dev/video0 -pix_fmt yuv420p -c:v libx264 -profile:v baseline -level:v 30 -b:v 800k -f mpegts - | avplay -

# Apple Devices from 2010 (-profile:v main -level 3.1)
ffmpeg -f video4linux2 -s 640x480 -r 30 -i /dev/video0 -pix_fmt yuv420p -c:v libx264 -profile:v high -level:v 3.1 -b:v 800k -f mpegts - | ffplay -
avconv -f video4linux2 -s 640x480 -r 30 -i /dev/video0 -pix_fmt yuv420p -c:v libx264 -profile:v high -level:v 31 -b:v 800k -f mpegts - | avplay -

# Apple Ipad2 and AppleTV 3 (-profile:v high -level 4.1)
ffmpeg -f video4linux2 -s 640x480 -r 30 -i /dev/video0 -pix_fmt yuv420p -c:v libx264 -profile:v baseline -level:v 41 -b:v 800k -f mpegts - | ffplay -
avconv -f video4linux2 -s 640x480 -r 30 -i /dev/video0 -pix_fmt yuv420p -c:v libx264 -profile:v baseline -level:v 41 -b:v 800k -f mpegts - | avplay -
\end{verbatim}

%}}}

\subsection{Quality control}
%{{{

\begin{itemize}

\item This mode compress the video using a constant quality along
  time. Therefore, it provides the maximun quality although we don't
  have control on the bit-rate (the only we know, as rule of thumb, is
  that every increase pf the quantization value by 5 halves the
  bitrate, 0 being lossless encoding).

\begin{verbatim}
ffmpeg -f video4linux2 -s 640x480 -r 30 -i /dev/video0 -c:v libx264 -crf 38 -crf_max 42 -f mpegts - | ffplay -
avconv -f video4linux2 -s 640x480 -r 30 -i /dev/video0 -c:v libx264 -crf 38 -crf_max 42 -f mpegts - | avplay -
\end{verbatim}
%avconv -f video4linux2 -s 640x480 -r 30 -i /dev/video0 -c:v libx264 -crf 38 -crf_max 42 -c:a aac out.mov

\item You can control the maximun and minimun bit-rates, and the
  maximum buffer size. In this case, the quantization value could
  change although his value can be also controlled. Example:

\begin{verbatim}
ffmpeg -f video4linux2 -s 640x480 -r 30 -i /dev/video0 -c:v libx264 -crf 38 -crf_max 42 -maxrate 4M -bufsize 2M -f mpegts - | ffplay -
avconv -f video4linux2 -s 640x480 -r 30 -i /dev/video0 -c:v libx264 -crf 38 -crf_max 42 -maxrate 4M -bufsize 2M -f mpegts - | avplay -
\end{verbatim}

\end{itemize}

%}}}

\subsection{Bit-rate control}
%{{{

\begin{itemize}

\item In order to precisely hit a defined filesize, thus a precise
  average bitrate, two-pass encoding is suggested. It takes more time
  than using crf and the result perceptually cannot be better, but in
  many scenarios the size requirement can be the most important issue.

\begin{verbatim}
time ffmpeg -i /tmp/coastguard_352x288x30x420x300.avi -c:v libx264 -preset medium -b:v 800k -pass 1 -f null -
time ffmpeg -y -i /tmp/coastguard_352x288x30x420x300.avi -c:v libx264 -preset medium -b:v 800k -pass 2 /tmp/out.mkv

time avconv -i /tmp/coastguard_352x288x30x420x300.avi -c:v libx264 -preset medium -b:v 800k -pass 1 -f null -
time avconv -y -i /tmp/coastguard_352x288x30x420x300.avi -c:v libx264 -preset medium -b:v 800k -pass 2 /tmp/out.mkv
\end{verbatim}

\end{itemize}

%}}}

\subsection{Lossless encoding}
%{{{

\begin{itemize}

\item Setting \texttt{-crf} or \texttt{-qp} to 0 forces x264 in
  lossless mode. In this case, the \texttt{-preset} settings then
  impact just the speed/size.

\end{itemize}

\begin{enumerate}
\item Fastest encoding:
\begin{verbatim}
avconv -i input -c:v libx264 -preset superfast -qp 0 -c:a copy out.mkv
\end{verbatim}

\item Smallest filesize:
\begin{verbatim}
avconv -i input -c:v libx264 -preset veryslow -qp 0 -c:a copy out.mkv
\end{verbatim}
\end{enumerate}

%}}}

%}}}

\section{HEVC}
\begin{verbatim}
# H.264 lossless:
wget http://www.hpca.ual.es/~vruiz/videos/coastguard_352x288x30x420x300.avi

# Lossy encode:
ffmpeg -i coastguard_352x288x30x420x300.avi -c:v hevc coastguard_352x288x30x420x300.mp4

# Get info:
ffprobe coastguard_352x288x30x420x300.mp4

# Play it:
ffprobe coastguard_352x288x30x420x300.mp4
\end{verbatim}

\section{\href{https://trac.ffmpeg.org/wiki/Stereoscopic}{3D}}
%{{{

\begin{verbatim}
# Download a SBS 3D video:
youtube-dl https://www.youtube.com/watch?v=YWWbZLl2evE

# Change the name:
mv Hang Gliding With The GoPro - 3D Side by Side (SBS)-YWWbZLl2evE.mp4 sbs.mp4

# Convert from SBS to "Red cyan gray/monochrome" anaglyph:
ffmpeg -i sbs.mp4 -vf stereo3d=sbs2l:arcg -acodec copy anaglyph.mp4
\end{verbatim}

%}}}

%}}}

\chapter{Editing}
%{{{

\section{Extracting and inserting images}
%{{{

\begin{verbatim}
wget http://www.hpca.ual.es/~vruiz/videos/coastguard_352x288x30x420x300.avi

# Extracting the images of a (color) video
ffmpeg -i coastguard_352x288x30x420x300.avi coastguard_%3d.png
avconv -i coastguard_352x288x30x420x300.avi coastguard_%3d.png
animate *.png

# Creating a video with a collection of images
ffmpeg -i coastguard_%3d.png coastguard.avi
avconv -i coastguard_%3d.png coastguard.avi
mplayer coastguard.avi
\end{verbatim}

%}}}

\section{Creating two videos in parallel}
%{{{

\begin{verbatim}
# Creating two videos (with two different formats) in parallel
ffmpeg -i coastguard_352x288x30x420x300.avi -map 0 1.mkv -map 0 2.mp4
ffmpeg -i 1.mkv
ffmpeg -i 2.mp4

avconv -i coastguard_352x288x30x420x300.avi -map 0 1.mkv -map 0 2.mp4
avconv -i 1.mkv
avconv -i 2.mp4
\end{verbatim}

%}}}

\section{Copy and paste}
%{{{

\begin{verbatim}
wget http://www.hpca.ual.es/~vruiz/videos/mobile_352x288x30x420x300.avi

ffmpeg -i coastguard_352x288x30x420x300.avi
ffmpeg -i mobile_352x288x30x420x300.avi

# Concatenate two videos
ffmpeg -i "concat:coastguard_352x288x30x420x300.avi|mobile_352x288x30x420x300.avi" -y -map 0 -c copy concat.avi
vlc concat.avi

# Also works for concatenatig:
# Create a text file containing a list with the files:
cat list.txt
file './coastguard_352x288x30x420x300.avi'
file './mobile_352x288x30x420x300.avi'

ffmpeg -f concat -safe 0 -i list.txt -c copy concat.mp4

# Paste two videos
ffmpeg -i coastguard_352x288x30x420x300.avi -i mobile_352x288x30x420x300.avi -c copy -map 0:0 -map 1:0 concat.avi
vlc concat.avi
\end{verbatim}

%}}}

%}}}

\chapter{Transcoding}
%{{{

\section{Bit-rate}
%{{{

\begin{verbatim}
# What's inside a file:
ffmpeg -i input.mkv

# Set the video bitrate of the output file to 64 kbit/s:
$ ffmpeg -i input.avi -b:v 64k -bufsize 32k output.avi
\end{verbatim}

%}}}

\section{Frame-rate}
%{{{

\begin{verbatim}
# Force the frame rate of the output file to 24 fps:
ffmpeg -i input.avi -r 24 output.avi

# Force the frame rate of the input file (valid for raw formats
# only) to 1 fps and the frame rate of the output file to 24 fps:
ffmpeg -r 1 -i input.m2v -r 24 output.avi
\end{verbatim}

%}}}

%}}}

\chapter{Streaming} % http://sonnati.wordpress.com/2011/08/30/ffmpeg-%E2%80%93-the-swiss-army-knife-of-internet-streaming-%E2%80%93-part-iv/
%{{{

\section{One-to-one unicast}
%{{{

\begin{verbatim}
# Sender:

## UDP:
ffmpeg -f alsa -i hw:0,0 -acodec mp2 -f video4linux2 -s 640x480 -r 30 -i /dev/video0 -f mpegts udp://localhost:8888 # Runme first!

## TCP:
ffmpeg -f video4linux2 -s 640x480 -r 30 -i /dev/video0 -f mpegts tcp://localhost:8888 # Runme last!

## RTP/RTSP
ffmpeg -f video4linux2 -s 640x480 -r 30 -i /dev/video0 -f rtsp -rtsp_transport tcp rtsp://localhost:8888/live.sdp # Runme last!

# Receiver:

## UDP:
ffplay -i udp://localhost:8888 # Runme last!

## TCP:
ffplay -i tcp://localhost:8888?listen # Runme first!

## RTP/RTSP
ffplay -rtsp_flags listen rtsp://localhost:8888/live.sdp?tcp # Runme first!
\end{verbatim}

%}}}

\section{Multicast}
%{{{

\begin{verbatim}
# Sender:

## UDP:
ffmpeg -f video4linux2 -s 640x480 -r 30 -i /dev/video0 -f mpegts udp://236.0.0.1:8888 # Runme first!

# Receiver:

## UDP:
ffplay -i udp://236.0.0.1:8888 # Runme last!
\end{verbatim}

%}}}

\section{One-to-many unicast}
%{{{ 

\begin{verbatim}
cat > ffserver.conf << EOF
Port 8090 
# bind to all IPs aliased or not 
BindAddress 0.0.0.0 
# max number of simultaneous clients 
MaxClients 1000 
# max bandwidth per-client (kb/s) 
MaxBandwidth 10000 
# Suppress that if you want to launch ffserver as a daemon. 
NoDaemon 

<Feed feed1.ffm> 
File /tmp/feed1.ffm 
FileMaxSize 5M 
</Feed> 

# FLV output - good for streaming 
<Stream test.flv> 
# the source feed 
Feed feed1.ffm 
# the output stream format - FLV = FLash Video 
Format flv 
VideoCodec flv 
# this must match the ffmpeg -r argument 
VideoFrameRate 15 
# generally leave this is a large number 
VideoBufferSize 80000 
# another quality tweak 
VideoBitRate 200 
# quality ranges - 1-31 (1 = best, 31 = worst) 
VideoQMin 1 
VideoQMax 5 
VideoSize 352x288 
# this sets how many seconds in past to start 
PreRoll 0 
# wecams don't have audio 
Noaudio 
</Stream> 

# ASF output - for windows media player 
<Stream test.asf> 
# the source feed 
Feed feed1.ffm 
# the output stream format - ASF 
Format asf 
VideoCodec msmpeg4 
# this must match the ffmpeg -r argument 
VideoFrameRate 15 
# generally leave this is a large number 
VideoBufferSize 80000 
# another quality tweak 
VideoBitRate 200 
# quality ranges - 1-31 (1 = best, 31 = worst) 
VideoQMin 1 
VideoQMax 5 
VideoSize 352x288 
# this sets how many seconds in past to start 
PreRoll 0 
# wecams don't have audio 
Noaudio 
</Stream>

##################################################################
# SDP/multicast examples
#
# If you want to send your stream in multicast, you must set the
# multicast address with MulticastAddress. The port and the TTL can
# also be set.
#
# An SDP file is automatically generated by ffserver by adding the
# 'sdp' extension to the stream name (here
# http://localhost:8090/test1-sdp.sdp). You should usually give this
# file to your player to play the stream.
#
# The 'NoLoop' option can be used to avoid looping when the stream is
# terminated.
<Stream test1-sdp.mpg>
Format rtp
Feed feed1.ffm 
#File "/usr/local/httpd/htdocs/test1.mpg"
MulticastAddress 224.124.0.1
MulticastPort 5000
MulticastTTL 16
NoLoop
</Stream>
EOF

xterm -e "ffserver -f ./ffserver.conf" &

xterm -e "ffmpeg -f alsa -i hw:0,0 -acodec mp2 -f video4linux2 -s 640x480 -r 30 -i /dev/video0 http://localhost:8090/feed1.ffm" &

mplayer http://localhost:8090/test.asf
\end{verbatim}

%}}}

\begin{comment}
\section{\href{https://trac.ffmpeg.org/wiki/Encode/FFV1}{Fast Forward Video (FFV)}}
%{{{

\begin{verbatim}
ffmpeg -codecs | grep ffv
...
 DEV..S ffv1                 FFmpeg video codec #1
 DEVI.S ffvhuff              Huffyuv FFmpeg variant

ffmpeg -f video4linux2 -s 640x480 -r 30 -i /dev/video0 -vcodec ffv1 -level 1 -coder 1 -context 1 -g 1 -f mpegts - | ffplay -
avconv -f x11grab -show_region 1 -follow_mouse centered -framerate 25 -video_size cif -i :0.0 -vcodec ffv1 -level 1 -coder 1 -context 1 -g 1 -f mpegts - | avplay -
\end{verbatim}

%}}}
\end{comment}



